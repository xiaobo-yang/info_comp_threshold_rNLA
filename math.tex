\chapter{Math Foundation}
Basic math tools.

%---------------------------------------------------
%---------------------------------------------------
\section{Key elements}
\subsection{Yao's principle}
When we analyze complexity of average-case problem with randomized algorithm, we can analyze a problem with randomized input and deterministic algorithm instead.
\begin{thm}[Yao's minimax principle\cite{yao_prin}]
 \label{yaoprin}
Given a problem $\mc P$, let $\mc A$ be the set of all deterministic algorithms to solve $\mc P$ and $\mc X$ be the set of all instances of $\mc P$.
Let $c(a,x)$ be the cost of algorithm $a\in\mc A$ solving instance $x\in\mc X$. For any distribution $p$ over $\mc A$ and any distribution $q$ over input space $\mc X$, consider randomized algorithm $A\sim p$ and random instance $X\sim q$. Then, we have
$$
\max _{x \in \mathcal{X}} \mathbf{E}[c(A, x)] \geq \min _{a \in \mathcal{A}} \mathbf{E}[c(a, X)]
$$
\end{thm}
That is, the worst-case expected cost of the randomized algorithm is at least the expected cost of the best deterministic algorithm against input distribution $q$.
\pf{\ref{yaoprin}}{
Let $C=\max _{x \in \mathcal{X}} \mathbf{E}[c(A, x)]$ and $D=\min _{a \in \mathcal{A}} \mathbf{E}[c(a, X)]$. We have
$$
\begin{aligned}
C & =\sum_x q_x C  \geq \sum_x q_x \mathbf{E}[c(A, x)]  =\sum_x q_x \sum_a p_a c(a, x)=\sum_a p_a \sum_x q_x c(a, x)  =\sum_a p_a \mathbf{E}[c(a, X)]  \geq \sum_a p_a D=D
\end{aligned}
$$
}

%---------------------------------------------------
\subsection{Model of computation}


%---------------------------------------------------
%---------------------------------------------------
\section{Concentration and anti-concentration inequalities}


%---------------------------------------------------
\subsection{Overlap of a vector on a large set}
If two subsets of $\R^n$ are "large", then the "overlap" of them is large. Here, "large" means the subsets have large meassure under Gaussian measure, and "overlap" means: For any non-empty $A,B\subseteq \R^n$. Denote by $\gamma_{|A\times B}$ the probability measure corresponding to the normalized restriction of $\mc N(0,I_n)\times \mc N(0,I_n)$ to $A\times B$ and let
\[
v(A,B):=\mb E_{(x,y)\sim\gamma_{|A\times B}}[|\langle x,y \rangle|^2].
\]
Formally, we have
\begin{thm}[\cite{vidick2012concentration}]
\label{large_overlap}
For any $\eta>0$, there exists ${ }^1$ a $\delta>0$ such that for all large enough $n$, if $A, B$ both have measure $\gamma(A), \gamma(B) \geq e^{-\delta n}$ then
$$
v(A, B) \geq(1-\eta) v\left(\mathbb{R}^n, \mathbb{R}^n\right)=(1-\eta) n
$$
\end{thm}
It's continuous version of \eqref{anti_concen_ghd}, which is more general to yield corruption bound for communication problem with continuous domain $\mc X,\mc Y$ and rectangle $R$.

There is an similar anti-concentration result in \cite{chakrabarti2012information}, where they study ORT problem.
\begin{thm}[Anti-concentration lemma 4.1 in \cite{chakrabarti2012information}]
\label{ort_anti_concen} 
For sufficiently large $n$ and $b\ge 66$, let $\epsilon=1-\Phi(2.01b)$. Then there exists $\delta>0$ 
such that for all $A,B\subseteq\{\pm1\}^n$ with $\min\{|A|,|B|\}\ge2^{n-\delta n}$, we have
\[
\mb P_{(X,Y)\sim Unif\{A\times B\}}[|\langle X,Y \rangle|>b\sqrt{n}]\ge \epsilon.    
\]
\end{thm}
\begin{rmk}
This lemma incorporates lemma 3.3 in \cite{sherstov2012communication}(key of theorem \ref{cor_bound}) as a speacial case of $b=1/4$.
Here $b$ is general but the analysis in \cite{sherstov2012communication} fails when $b>1$.
\end{rmk}
%---------------------------------------------------
\subsection{Hutchinson's trace estimator}
A simple estimator of trace of matrix by monte carlo method and we use only matrix-vector product oracle.
\begin{defn}[Hutchinson's Estimator]
\label{hutchinson_est}
Given a matrix $A\in\R^{d\times d}$. We estimate $tr(A)$ by
\[
H_m(A) = \frac{1}{m}\sum_{i=1}^m g_i^T A g_i = \frac{1}{m}tr(G^T A G),
\]
where $G=[g_1,...,g_m]\in\R^{d\times m}$ is a Gaussian matrix with $i.i.d.$ $\mc N(0,1)$ entries.
\end{defn}

\begin{thm}[Hutchinson analysis\cite{mmmw21}]
\label{hutchinson_bound}
Let $A \in \mathbb{R}^{d \times d}, \delta \in(0,1 / 2], \ell \in \mathbb{N}$. Let $\mathrm{H}_{\ell}(A)$ be the $\ell$-query Hutchinson estimator defined above, implemented with mean 0, i.i.d. sub-Gaussian random variables with constant sub-Gaussian parameter. For fixed constants $c, C$, if $\ell>c \log (1 / \delta)$, then with probability $\geq 1-\delta$,
$$
\left|\mathrm{H}_{\ell}(A)-\operatorname{tr}(A)\right| \leq C \sqrt{\frac{\log (1 / \delta)}{\ell}}\|A\|_F .
$$
So, if $\ell=O\left(\frac{\log (1 / \delta)}{\varepsilon^2}\right)$ then, with probability $\geq 1-\delta,\left|\mathrm{H}_{\ell}(A)-\operatorname{tr}(A)\right| \leq \varepsilon\|A\|_F$.
\end{thm}
\pfsk{\ref{hutchinson_bound}}{
First, vectorize gaussian matrix $G$ to $\bar{g}\in\R^{dl}$ and let $\bar{A}=diag\{A,A,...,A\}\in\R^{dl\times dl}$.

Then, $H_l(A)=\bar{g}^T\bar{A}\bar{g}/l$ and use Hanson-Wright inequality\cite{hdp}.
}

%---------------------------------------------------
\subsection{Talagrand's inequality}
\begin{thm}[Talagrand\cite{sherstov2012communication}]
\label{tal_ineq}
For a fixed convex set $S \subseteq \mathbb{R}^n$ and a random $x \in\{-1,+1\}^n$,
$$
\mathbb{P}[x \in S] \mathbb{P}[\rho(x, S)>t] \leq \mathrm{e}^{-t^2 / 16},
$$
where $\rho(x, S)=\inf _{y \in S}\|x-y\|$.
\end{thm}
Intuitively, it means measure concentration around convex hull of large subset $S\subseteq\{\pm 1\}^n$, where the measure means counting measure(or uniform distribution). It can be seen as an isoperimetric inequality in discrete structures.

The usefulness of Talagrand's inequality is led by appropriately choosing set $S$, according to property we interest. Here are some useful corollaries.
\begin{cor}[\cite{sherstov2012communication}]
\label{tal_ineq_proj}
For every linear subspace $V \subseteq \mathbb{R}^n$ and every $t>0$, one has
$$
\underset{x \in\{-1,+1\}^n}{\mb{P}}\left[\left|\left\|\operatorname{proj}_V x\right\|-\sqrt{\operatorname{dim} V}\right|>t\right]<4 \mathrm{e}^{-c t^2},
$$
where $c>0$ is an absolute constant.
\end{cor}
\pf{\ref{tal_ineq_proj}}{ First, note that it suffices to prove
$$
\mathbf{P}[|\rho(x, V)-\sqrt{n-\operatorname{dim} V}|>t] \leq 4 \mathrm{e}^{-\Omega\left(t^2\right)}
$$
and $n-\operatorname{dim} V= tr(\mb E[x^T(I-P_V)x]) =\mb E[\rho(x,V)^2]$, where $P_v$ is orthogonal projection on $V$.
 Therefore, take $S=\{v\in\R^n:\rho(v,V)\le a\}$, then by Talagrand's inequality we have
\[
\mathbb{P}[\rho(x, V) \leq a] \mathbb{P}[\rho(x, V)>a+t] \le \mathbb{P}[\rho(x, V) \leq a] \mathbb{P}[\rho(x, S)>t]  \leq \mathrm{e}^{-t^2 / 16}.
\]
Then, take $a=median(\rho(x,V)):=m$, we have
$$
\begin{aligned}
& \mathbb{P}[\rho(x, V)>m+t] \leq 2 \mathrm{e}^{-t^2 / 16}, \\
& \mathbb{P}[\rho(x, V) \leq m-t] \leq 2 \mathrm{e}^{-t^2 / 16}.
\end{aligned}
$$
Finally, the result follows $m=\sqrt{\mb E[\rho(x,V)^2]}+O(1)$. See \cite{tao09web} for details.
}
%---------------------------------------------------
\subsection{Sketching}
\begin{defn}[projection-cost-preserving\cite{musco2020projectioncostpreserving}]
\label{proj_preserve_def}
A matrix $\tilde{A}\in\rnm nm$ is an $(\epsilon,c,k)$ projection-cost-preserving sketching of $A\in\rnm nd$ if for any orthogonal projection $P\in\rn n$ with rank at most $k$, we have
\[(1-\epsilon)\|A-PA\|_F^2\le\|\tilde{A}-P\tilde{A}\|_F^2+c\le (1+\epsilon)\|A-PA\|_F^2.\]
\end{defn}

\begin{thm}[simple sketching is projection-cost-preserving\cite{musco2020projectioncostpreserving}]
    \label{proj_preserve_sk}
    Let $S\in\rnm dm$ be a $i.i.d.$ standard Gaussian matrix. If $m\ge c(k+\log(1/\delta))/\epsilon^2$ for some large constant $c$, we have $\frac{1}{\sqrt{m}}AS$
    is a $(\epsilon,0,k)$-projection-cost-preserving sketching of $A$ with probability at least $1-\delta$.
\end{thm}
\pfsk{\ref{proj_preserve_sk}}{

}

\begin{thm}[projection-cost-preserving makes projection approximation\cite{musco2020projectioncostpreserving}]
    \label{proj_preserve_thm}
    If  $\tilde{A}\in\rnm nm$ is an $(\epsilon,c,k)$ projection-cost-preserving sketching of $A\in\rnm nd$, then for any subset $\mc T$ of all
    orthogonal projection with rank at most $k$, if we take a $\tilde{P}\in\mc T$ such that $\|\tilde{A}-\tilde{P}\tilde{A}\|_F^2\le\gamma\min_{P\in\mc T}\|\td A-P\td A\|_F^2$,
    we have
    \[
    \fnorm{A-\td P A}^2\le \frac{1+\epsilon}{1-\epsilon}\gamma\min_{P\in\mc T}\fnorm{A-PA}^2+\frac{(1-\gamma)c}{1-\epsilon}.    
    \]
\end{thm}
\pf{\ref{proj_preserve_thm}}{Direct calculation.}

\begin{thm}[(Upper Bound on Regression Error\cite{jpwz21}]
    \label{regress_sketch}
    Given $\delta\in(0,1/2)$ and matrices $A,B$ with $n$ rows and $rank(A)=k$. Let $S\in\rnm rn$ be a random matrix with $i.i.d.$ Gaussian $\mc N(0,\frac{1}{r})$.
    Let $\widetilde{X}=\arg\min_X\fnorm{S(AX-B)}$ and $X^*=\arg\min_X\fnorm{AX-B}$. Then, if $r=\Omega(k+\log(1/\delta))$, we have with probability at least $1-\delta$
    \[
    \fnorm{A\wtd{X}-B}\le O(1)\fnorm{AX^*-B}.    
    \]
\end{thm}
\begin{rmk}
\cite{cw09} shows what $O(1)$ is, while the randomness there is rademacher's variable.
\end{rmk}

\pfsk{\ref{regress_sketch}}{

}

%---------------------------------------------------
%---------------------------------------------------
\section{Statistical distance}
\begin{thm}[\cite{pwbm18} proposition 5.11]
\label{spiked_gaussian_chi_dist}
For any $|\beta|<1$, there exists $\delta>0$ such that the following holds. Let $\mc X=\{\mc X_n\}$ be a family of prior distribution of spiked vectors $x$ with $1-\delta\le\|x\|\le 1+\delta$. Let $Q_n$ be joint distribution of $N$ $i.i.d$ samples from $\mc N(0,I_n+\beta xx^T)$ and $P_n$ be joint distribution of $N$ $i.i.d$ samples from $\mc N(0,I_n)$. Then we have
\[
\mb E_{P_n}\bbfr{\bbr{\frac{dQ_n}{dP_n}}^2} = \mb E_{x,x'\sim\mc X_n}\bbfr{(1-\beta^2\langle x,x' \rangle^2)^{-N/2}}.
\]
\end{thm}
\pf{\ref{spiked_gaussian_chi_dist}}{
First, expand LHS directly, we have $ \underset{P_n}{\mathbb{E}}\left[\left(\frac{\mathrm{d} Q_n}{\mathrm{~d} P_n}\right)^2\right]=\underset{Q_n}{\mathbb{E}}\left[\frac{\mathrm{d} Q_n}{\mathrm{~d} P_n}\right]$
$$
\begin{aligned}
& \frac{\mathrm{d} Q_n}{\mathrm{~d} P_n}\left(y_1, \ldots, y_N\right)=\underset{x^{\prime} \sim \mathcal{X}}{\mathbb{E}}\left[\prod_{i=1}^n \frac{\exp \left(-\frac{1}{2} y_i^{\top}\left(I+\beta x^{\prime} x^{\prime \top}\right)^{-1} y_i\right)}{\sqrt{\operatorname{det}\left(I+\beta x^{\prime} x^{\prime \top}\right)} \exp \left(-\frac{1}{2} y_i^{\top} y_i\right)}\right] \\
& =\underset{x^{\prime}}{\mathbb{E}}\left[\operatorname{det}\left(I+\beta x^{\prime} x^{\prime \top}\right)^{-N / 2} \prod_{i=1}^N \exp \left(-\frac{1}{2} y_i^{\top}\left(\left(I+\beta x^{\prime} x^{\prime \top}\right)^{-1}-I\right) y_i\right)\right] .
\end{aligned}
$$
Then, simplify it by  Sherman–Morrison formula, we have $\left(I+\beta x^{\prime} x^{\prime \top}\right)^{-1}-I=\frac{-\beta}{1+\beta\left\|x^{\prime}\right\|^2} x^{\prime} x^{\prime \top}$, and then
$$
=\underset{x^{\prime}}{\mathbb{E}}\left[\left(1+\beta\left\|x^{\prime}\right\|^2\right)^{-N / 2} \prod_{i=1}^N \exp \left(\frac{1}{2} \frac{\beta}{1+\beta\left\|x^{\prime}\right\|^2}\left\langle y_i, x^{\prime}\right\rangle^2\right)\right] .
$$
Finally, passing to the second moment, we compute
$$
\begin{aligned}
& \underset{P_n}{\mathbb{E}}\left[\left(\frac{\mathrm{d} Q_n}{\mathrm{~d} P_n}\right)^2\right]=\underset{x, x^{\prime}}{\mathbb{E}}\left[\left(1+\beta\left\|x^{\prime}\right\|^2\right)^{-N / 2} \prod_{i=1}^N \underset{y_i \sim \mathcal{N}\left(0, I+\beta x x^{\top}\right)}{\mathbb{E}} \exp \left(\frac{1}{2} \frac{\beta}{1+\beta\left\|x^{\prime}\right\|^2}\left\langle y_i, x^{\prime}\right\rangle^2\right)\right]\\
& =\underset{x, x^{\prime}}{\mathbb{E}}\left[\left(1+\beta\left\|x^{\prime}\right\|^2\right)^{-N / 2} \prod_{i=1}^N\left(1-\frac{\beta}{1+\beta\left\|x^{\prime}\right\|^2}\left(\left\|x^{\prime}\right\|^2+\beta\left\langle x, x^{\prime}\right\rangle^2\right)\right)^{-1 / 2}\right] \\
& =\underset{x, x^{\prime}}{\mathbb{E}}\left[\left(1-\beta^2\left\langle x, x^{\prime}\right\rangle^2\right)^{-N / 2}\right]
\end{aligned}
$$
Note that, the condition about $\delta$ is because here the MGF step requires
$
\frac{\beta}{1+\beta\left\|x^{\prime}\right\|^2}\left(\left\|x^{\prime}\right\|^2+\beta\left\langle x, x^{\prime}\right\rangle^2\right)<1
$.
}

\begin{thm}[\cite{jpwz21}]
\label{gauss_kl}
For distribution $P=\mc N_k(\mu_1,\Sigma_1)$ and $Q=\mc N_k(\mu_2,\Sigma_2)$, we have
\[
d_{KL}(P,Q)=\frac{1}{2}\bbwr{
(\mu_2-\mu_1)^T\Sigma_2^{-1}(\mu_2-\mu_1)
+ tr(\Sigma_2^{-1}\Sigma_1) - \log\frac{det(\Sigma_1)}{det(\Sigma_2)-k}
}.
\]
\end{thm}
\pfsk{\ref{gauss_kl}}{}

\begin{thm}[Data processing inequality\cite{jpwz21}]
\label{data_process}
For random variable $X,Y$ and any function $f$, we have
\[
d_{KL}(f(X),f(Y))\le d_{KL}(X,Y).
\]
\end{thm}
\pfsk{\ref{data_process}}{}
\begin{rmk}
    There are also generalized data processing inequality for $f$-divergence.
    \xb{see chenkun's talk about\cite{simchowitz2020tight}}
\end{rmk}

%---------------------------------------------------
%---------------------------------------------------
\section{Random matrices}

%---------------------------------------------------
\subsection{Non-asymptotic theory}
In this section, we denote the distribution of random matrices $G\in\R^{n\times n}$ with $i.i.d.$ $\mc N(0,1)$ entries by $\mc N(n)$.
\begin{thm}[Remaining randomness\cite{simchowitz2020tight}\cite{jpwz21}]
\label{block_conditional} 
Let $G\sim\mc N(n)$. Let $W=(G+G^T)/2$. For any sequence of vector queries $v_1,...,v_T$, along with oracles $w_i=Wv_i$. Then, conditioned on these observations, there exists a rotation matrix $U$, independent of $w_i$, such that
\[
UWU^T=\begin{bmatrix}
    Y_1 & Y_2^T \\
    Y_2 & \widetilde{W}
\end{bmatrix},
\]
where $Y_1,Y_2$ are deterministic and $\widetilde{W}=(\widetilde{G}+\widetilde{G}^T)/2$, where $\widetilde{G}\sim\mc N(n-T)$.
\end{thm}
\pfsk{\ref{block_conditional}}{

}

\begin{thm}[concentration of the largest singular value\cite{jpwz21}]
\label{rmt_smax_concen}
Let $G\sim\mc N(n)$. Then, for any $t\ge 0$ we have
\[
\mb P[s_{max}(G)\le 2\sqrt{n}+t]\ge 1-2e^{-t^2/2}.
\]
\end{thm}
\pfsk{\ref{rmt_smax_concen}}{
    
}



%---------------------------------------------------
\subsection{Asymptotic theory}

%---------------------------------------------------
%---------------------------------------------------
\section{Matrix analysis}
\begin{lem}[\cite{mmmw21}]
    \label{lra_tr_bound}
For any PSD matrix $A$, we have $\|A-A_k\|_F\le\frac{tr(A)}{\sqrt{k}}$.  
\end{lem}
\pfm{lemma}{\ref{lra_tr_bound}}{
$LHS^2 = \sum_{i=k+1}^n\lambda_i^2\le\frac{tr(A)}{k} \sum_{i=k+1}^n\lambda_i\le\frac{tr(A)^2}{k}.$
}

\begin{lem}
\label{proj}
Given matrices $A,B$, we have $\arg\min_X\snorm{AX-B}=\arg\min_X\fnorm{AX-B}=A^+B$ and  $\arg\min_Y\snorm{YA-B}=\arg\min_Y\fnorm{YA-B}=BA^+$,
where $M^+$ is Moore–Penrose inverse of matrix $M$.
\end{lem}
\pf{\ref{proj}}{
\xb{see wiki of Moore–Penrose inverse, which shows $AA^+,A^+A$ are both orthogonal projections}
}
