\usepackage{algorithm} % algorithm block
\usepackage{algpseudocode}
\algrenewcommand\algorithmicrequire{\textbf{Input:}} % change REQUIRE to Input
\algrenewcommand\algorithmicensure{\textbf{Output:}}

\usepackage{booktabs} %pandas to latex
\usepackage{graphicx} %\resizebox

\usepackage[dvipsnames]{xcolor}
\usepackage{amsmath}
\usepackage{amssymb}
\usepackage{amsthm} % proof section
\usepackage{indentfirst}  %indent in the first line
\setlength{\parindent}{2em}

\usepackage{algorithm}
\usepackage{algorithmicx}
\usepackage{algpseudocode}
\renewcommand{\algorithmicrequire}{\textbf{Input:}} % Use Input in the format of Algorithm
\renewcommand{\algorithmicensure}{\textbf{Output:}} % Use Output in the format of Algorithm
\algnewcommand{\algorithmicand}{\textbf{and }}
\algnewcommand{\algorithmicor}{\textbf{or }}
\algnewcommand{\OR}{\algorithmicor}
\algnewcommand{\AND}{\algorithmicand}
\makeatletter
\newenvironment{breakablealgorithm}
  {% \begin{breakablealgorithm}
  \begin{center}
     \refstepcounter{algorithm}% New algorithm
     \hrule height.8pt depth0pt \kern2pt% \@fs@pre for \@fs@ruled
     \renewcommand{\caption}[2][\relax]{% Make a new \caption
      {\raggedright\textbf{\ALG@name~\thealgorithm} ##2\par}%
      \ifx\relax##1\relax % #1 is \relax
         \addcontentsline{loa}{algorithm}{\protect\numberline{\thealgorithm}##2}%
      \else % #1 is not \relax
         \addcontentsline{loa}{algorithm}{\protect\numberline{\thealgorithm}##1}%
      \fi
      \kern2pt\hrule\kern2pt
     }
  }{% \end{breakablealgorithm}
     \kern2pt\hrule\relax% \@fs@post for \@fs@ruled
  \end{center}
  }

\usepackage{mathrsfs} %for \mathscr

\usepackage{parskip} % paraskip

\usepackage[colorlinks=true,linkcolor=blue,citecolor=red,urlcolor=magenta]{hyperref}  % hyperlink

\newtheorem{thm}{Theorem}[section]
\newtheorem{pp}{Principle}[section]
\newtheorem{lem}{Lemma}[section]
\newtheorem{cor}{Corollary}[section]
\newtheorem{defn}{Definition}[section]
\newtheorem{rmk}{Remark}[section]

% proof
\newcommand{\pf}[2]{\begin{proof}[Proof for Theorem~#1]#2\end{proof}}
\newcommand{\pfsk}[2]{\begin{proof}[Proof sketch for Theorem~#1]#2\end{proof}}
\newcommand{\pfm}[3]{\begin{proof}[Proof for #1~#2]#3\end{proof}}
\newcommand{\pfskm}[3]{\begin{proof}[Proof sketch for #1~#2]#3\end{proof}}

% thm enviroment with feastures
\usepackage{tcolorbox}
% \newenvironment{blackboxtheorem}[1][]{\begin{tcolorbox}[colback=white,colframe=black,title=Problem:~#1]}{\end{tcolorbox}} % generate a blackbox environment
\newtcolorbox{problembox}[1][]{}  % generate a problem section with black box
\newtheorem{prob}{Problem}[section] % Define prob with section number
\numberwithin{prob}{section} % Number problems within sections
\tcolorboxenvironment{prob}{colback=white,colframe=black,fonttitle=\bfseries}


\newcommand{\PP}{\mathbb{P}}
\newcommand{\E}{\mathbb{E}}
\newcommand{\R}{\mathbb{R}}
\newcommand{\fnorm}[1]{\|#1\|_F}
\newcommand{\snorm}[1]{\|#1\|}
\newcommand{\xb}[1]{\textcolor{OliveGreen}{xiaobo: #1}}
\newcommand{\addtba}{\tba{(tba)}}
\newcommand{\mc}[1]{\mathcal{#1}}
\newcommand{\mb}[1]{\mathbb{#1}}
\newcommand{\mf}[1]{\mathbf{#1}}
\newcommand{\rn}[1]{\R^{#1\times #1}}
\newcommand{\rnm}[2]{\R^{#1\times #2}}


% big brackets
\newcommand{\bbr}[1]{\left(#1\right)}
\newcommand{\bbfr}[1]{\left[#1\right]}
\newcommand{\bbwr}[1]{\left\{#1\right\}}

% statistics
\newcommand{\indep}{\perp \!\!\! \perp}